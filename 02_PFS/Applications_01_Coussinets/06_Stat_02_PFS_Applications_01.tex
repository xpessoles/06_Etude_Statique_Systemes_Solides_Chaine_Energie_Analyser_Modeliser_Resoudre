\documentclass[10pt]{article}
\input{style/coursHeadings}
\input{style/programHeadings}
\input{style/macros_SII}
\input{style/macros_Titres}
\input{style/macros_Frames}

%Si le boolen xp est vrai : compilation pour xabi
%Sinon compilation Damien
\newboolean{xp}
\setboolean{xp}{true}

\newboolean{prof}
\setboolean{prof}{false}

\usepackage[%
    pdftitle={CI 06 : Stat - Modélisation des AM},
    pdfauthor={Xavier Pessoles},
    colorlinks=true,
    linkcolor=blue,
    citecolor=magenta]{hyperref}


\def\discipline{Sciences Industrielles de l'Ingénieur}
\def\xxtitre{\ifthenelse{\boolean{xp}}{
CI 06 : Étude du comportement statique des systèmes}{}}

\def\xxsoustitre{\ifthenelse{\boolean{xp}}{
Chapitre 1 -- Modélisation des Actions Mécaniques}{
Partie  -- }}

\def\xxauteur{\ifthenelse{\boolean{xp}}{
%Xavier \textsc{Pessoles}
}{}}

\def\xxpied{\ifthenelse{\boolean{xp}}{
CI 06 : Statique\\
Ch. 2 : PFS -- Application -- Coussinets}{
\xxtitre}}

\def\xxcathegorie{\ifthenelse{\boolean{xp}}{
2013 -- 2014 \\
Xavier \textsc{Pessoles}}{}}

%---------------------------------------------------------------------------

\begin{document}

\ifthenelse{\boolean{xp}}{\input{style/enteteXP}}{\input{style/enteteDI}}

\begin{center}
\Large{\textsc{Exercices d'applications : Choix de coussinet}}
\end{center}

\vspace{.25cm}
%
%\subsection*{Question de cours}
%\subparagraph*{}
%\textit{Après avoir donné leur schéma cinématique paramétré, donner les torseurs des actions mécaniques des liaisons suivantes : pivot d'axe $(A,\vect{y})$, appui plan de normale $\vect{x}$, glissière de direction $\vect{x}$.} 
%

\subsection*{Choix de coussinets -- Engrenage à denture droite}

\begin{obj}
\begin{itemize}
\item Déterminer le couple moteur nécessaire.
\item Choisir les coussinets.
\end{itemize}
\end{obj}


Le schéma d'architecture du montage ci-dessous présente un pignon arbré entraîné par \textbf{un moteur} (non représenté) de couple $C_m$.. La fréquence de rotation de l'arbre est de 180 tr/min.
L'effort à transmettre par l'engrenage est de 2400N. L'angle de pression dans l'engrenage est de 20\textdegree. On a $\vect{AB} = 3a \vect{z}$ et $\vect{AC} = a\vect{z}  + R \vect{y}$.

\begin{center}
\includegraphics[width=.7\textwidth]{images/modele}
\end{center}

\subsubsection*{Liaisons équivalentes -- Hors programme de colle}

\subparagraph{}
\textit{Tracer le graphe des liaisons.}

\subparagraph{}
\textit{Donner l'expression des torseurs statiques en $A$ et en $B$.}


\subparagraph{}
\textit{Déterminer l'expression du torseur statique de la liaison équivalente. De quelle liaison s'agit-il ?}

\subsubsection*{Détermination des actions mécaniques}

\subparagraph{}
\textit{Déterminer les actions mécaniques dans les liaisons A et B.}

%\subparagraph{}
%\textit{Isoler l'arbre et faire le bilan des actions mécaniques.}
\ifthenelse{\boolean{prof}}{
\begin{corrige}
Action mécanique exercée par la liaison rotule au point $A$ :
$$
\torseurstat{T}{0}{1_A}
= \torseurcol{X_A}{Y_A}{Z_A}{0}{0}{0}{A}
$$

Action mécanique exercée par la liaison linéaire annulaire au point $B$ :
$$
\torseurstat{T}{0}{1_B}
= \torseurcol{X_B}{Y_B}{0}{0}{0}{0}{B}
= \torseurcol{X_B}{Y_B}{0}{-3aY_B}{3aX_B}{0}{A}
$$

On a :
$$\vectm{A}{0}{1_B}=\vectm{B}{0}{1_B}+\vect{AB}\wedge \vectf{0}{1_B} 
= 3a\vect{z} \wedge \left(X_B\vect{x}+Y_B\vect{y} \right)
= 3aX_B \vect{y} - 3aY_B \vect{x}
$$

Action mécanique exercée par le couple moteur autour de $\vect{z}$ en $A$ :
$$
\torseurstat{T}{\text{moteur}}{1}
= \torseurcol{0}{0}{0}{0}{0}{C_m}{A}
$$


Action mécanique exercée par le pignon en $C$ :
$$
\torseurstat{T}{\text{pignon}}{1}
= \torseurcol{F\cos\alpha}{-F\sin\alpha}{0}{0}{0}{0}{C}
= \torseurcol{F\cos\alpha}{-F\sin\alpha}{0}{aF\sin\alpha}{aF\cos\alpha}{- RF\cos\alpha}{A}
$$

On a :
\begin{eqnarray*}
\vectm{A}{\text{pignon}}{1}
& =&\vectm{C}{\text{pignon}}{1}+\vect{AC}\wedge \vectf{\text{pignon}}{1} 
= \left(a\vect{z} + R\vect{y} \right) \wedge \left(F\cos\alpha\vect{x}-F\sin\alpha\vect{y} \right) \\
&= & aF\cos\alpha \vect{y} +aF\sin\alpha \vect{x} - RF\cos\alpha\vect{z}
\end{eqnarray*}

\end{corrige}
}{}
%\subparagraph{}
%\textit{Appliquer le PFS.}
\ifthenelse{\boolean{prof}}{
\begin{corrige}
On applique le PFS au solide 1 et on a :
$$
\torseurstat{T}{0}{1_B}  + \torseurstat{T}{0}{1_A} +\torseurstat{T}{\text{moteur}}{1} + \torseurstat{T}{\text{pignon}}{1} 
=\{0\}
$$

\end{corrige}
}{}

%\subparagraph{}
%\textit{Faire le bilan des inconnues et des équations. Déterminer les actions mécaniques dans les liaisons ainsi que le couple moteur nécessaire.}
\ifthenelse{\boolean{prof}}{
\begin{corrige}

Il y a 6 inconnues et 6 équations. A priori, il est donc possible de résoudre le système.

On a alors : 
$$
\torseurcol{X_A}{Y_A}{Z_A}{0}{0}{0}{A}
+ \torseurcol{X_B}{Y_B}{0}{-3aY_B}{3aX_B}{0}{A}
+ \torseurcol{0}{0}{0}{0}{0}{C_m}{A}
+ \torseurcol{F\cos\alpha}{-F\sin\alpha}{0}{aF\sin\alpha}{aF\cos\alpha}{- RF\cos\alpha}{A}
=\torseurcol{0}{0}{0}{0}{0}{0}{A}
$$
D'où :
$$
\left\{
\begin{array}{l}
X_A + X_B  + 0 +F\cos\alpha =0 \\
Y_A + Y_B  + 0 - F\sin\alpha =0 \\
Z_A + 0  + 0 + 0=0 \\
0    -3aY_B  + 0 + aF\sin\alpha=0 \\
0    +3aX_B + 0 + aF\cos\alpha=0 \\
0    + 0       + C_m -RF\cos\alpha=0 \\
\end{array}
\right.
\Longleftrightarrow
\left\{
\begin{array}{l}
X_A = - X_B  - F\cos\alpha =\dfrac{1}{3} F\cos\alpha- F\cos\alpha = -\dfrac{2}{3} F\cos\alpha\\
Y_A =  F\sin\alpha -Y_B =F\sin\alpha - \dfrac{1}{3}F\sin\alpha = \dfrac{2}{3}F\sin\alpha \\
Z_A =0 \\
Y_B   =\dfrac{1}{3}F\sin\alpha \\
X_B = -\dfrac{1}{3} F\cos\alpha\\
C_m  = RF\cos\alpha
\end{array}
\right.
$$
\textit{Application numérique :}
$$
\left\{
\begin{array}{l}
X_A = -1504 \;N\\
Y_A = 547 \;N \\
Z_A =0 \\
Y_B   =274 \; N\\
X_B = -752 \;N\\
C_m  = 2\, 255\; R\;Nm
\end{array}
\right.
$$
\end{corrige}
}{}
\subparagraph{}
\textit{Aurait-il été possible de trouver les actions mécaniques plus rapidement ?}

\newpage

\subsubsection*{Choix des coussinets}
Un calcul en torsion de l'arbre impose un diamètre supérieur à 30 mm. 
Les dimensions du mécanisme imposent $a=50\; mm$.

On désire savoir si un certain type de coussinet pourrait être utilisé :
\begin{itemize}
\item diamètre nominal intérieur du coussinet : 35 mm;
\item diamètre nominal extérieur du coussinet : 39 mm;
\item longueurs de coussinet disponibles : 20, 30, 35 et 50 mm.
\end{itemize}

La pression maximale admissible entre l'arbre et le coussinet est de 14 MPa. Le constructeur impose que le produit pV (pression multipliée par la vitesse) soit inférieur à 0,7 Mpa m/s.

\subparagraph{}
\textit{Est-il possible d'utiliser un des coussinets proposés ?}
\ifthenelse{\boolean{prof}}{
\begin{corrige}
Évaluation des efforts radiaux: 
\begin{itemize}
\item effort radial dans le coussinet $A$ : $F_{rA} = \sqrt{X_A^2 + Y_A^2} \simeq 1600 \;  N$;
\item effort radial dans le coussinet $B$ : $F_{rB} = \sqrt{X_B^2 + Y_B^2} \simeq 800 \; N$.
\end{itemize}

Évaluation des pressions de contacts : 
\begin{itemize}
\item pression de contact dans le coussinet $A$ : $p_{A} = \dfrac{F_{rA}}{d_a\cdot L_A}$;
\item pression de contact dans le coussinet $B$ : $p_{B} = \dfrac{F_{rB}}{d_a\cdot L_B}$.
\end{itemize}
On souhaite que $p_{A}$ et $p_{B}$ soient inférieurs à 14 MPa. On a donc :
\begin{itemize}
\item $\dfrac{F_{rA}}{d_a\cdot L_A} < 14 \Longrightarrow L_A > \dfrac{F_{rA}}{d_a\cdot 14}\simeq 3,3\; mm$;
\item $\dfrac{F_{rB}}{d_a\cdot L_B} < 14 \Longrightarrow L_B > \dfrac{F_{rB}}{d_a\cdot 14}\simeq 1,6 \; mm$.
\end{itemize}

Par ailleurs, le critère $pV$ impose que 
$$
pV<0,7 \Longleftrightarrow \dfrac{F}{d_a L}\cdot \dfrac{2\pi N}{60} \dfrac{d_a}{2} <0,7 
\Longleftrightarrow L>\dfrac{F\pi N}{0,7\cdot 60}
\Longleftrightarrow L>4,28F
$$
Sauf erreur, les coussinets choisis ne conviendront pas.

\end{corrige}
}

%\newpage

\subsection*{Choix de coussinets -- Engrenage à denture hélioïdale}
\setcounter{subparagraph}{0}
\begin{obj}
\begin{itemize}
\item Déterminer le couple moteur nécessaire.
\item Choisir les coussinets.
\end{itemize}
\end{obj}


Le schéma d'architecture du montage ci-dessous présente un pignon conique relié à un arbre,  entraîné par un moteur (non représenté). La fréquence de rotation de l'arbre est de 120 tr/min.

La résultante des efforts au point $A$ est donné par $\vect{F_A} = -10\,000\vect{x} +3200\vect{y} + 1700\vect{z}$.

\begin{center}
\includegraphics[width=.75\textwidth]{images/modele2}
\end{center}

\subsubsection*{Détermination des actions mécaniques}

\subparagraph{}
\textit{Déterminer les actions mécaniques dans les paliers en $B$ et $C$.}

\subsection*{Choix des coussinets}
Le concepteur du montage impose que le rapport entre la longueur de guidage et le diamètre du coussinet soit compris entre 0,6 et 0,8.

Un constructeur de coussinets donne les conditions de fonctionnement suivante : 
\begin{itemize}
\item $p_{max}=18 \; MPa$;
\item $V_{maxi}= 6\; m\cdot s^{-1}$;
\item $(pV)_{maxi} = 1,8 MPa \; \cdot m\cdot s^{-1}$.
\end{itemize}

\subparagraph{}
\textit{Déterminer les dimensions d'un coussinet à collerette en $B$ et d'un coussinet sans collerette en $C$.}


\end{document}


