\documentclass[10pt,fleqn]{article} % Default font size and left-justified equations
\usepackage[%
    pdftitle={CIN : Vérification des performances cinématiques des systèmes},
    pdfauthor={Xavier Pessoles}]{hyperref}
    
\input{style/new_style}
\input{style/macros_SII}

\usepackage{multicol}
\fichetrue
%\fichefalse

%\proftrue
\proffalse

\tdtrue
%\tdfalse

\courstrue
\coursfalse

\def\discipline{Sciences \\Industrielles de \\ l'Ingénieur}
\def\xxtete{Sciences Industrielles de l'Ingénieur}

\def\classe{PTSI}
\def\xxnumpartie{Cycle 9}
\def\xxpartie{Modélisation des actions mécaniques\\
Modéliser}

\def\xxnumchapitre{Chapitre 1}
\def\xxchapitre{Modélisation des actions de pesanteur et de contact sans frottement}

\def\xxtitreexo{Exercices divers}
\def\xxsourceexo{\hspace{.2cm} Sources diverses.}


\def\xxposongletx{2}
\def\xxposonglettext{1.45}
\def\xxposonglety{20}
\def\xxonglet{Cycle 9 -- Ch. 1}

\def\xxactivite{Colle 1}
\def\xxauteur{\textsl{Xavier Pessoles}}

\def\xxcompetences{%
\textsl{%
%\textbf{Savoirs et compétences :}\\
%\noindent \textbf{Analyser :} 
%\begin{itemize}[label=\ding{112},font=\color{ocre}] 
%\item \textit{A3 -- C6 :} transmetteurs de puissance.
%\end{itemize}
%\noindent \textbf{Modéliser :} \textit{proposer un modèle de connaissance du système.}
}}

\def\xxfigures{
%\includegraphics[width=.7\textwidth]{images/broyeur_01}
}%figues de la page de garde

\def\xxpied{%
Cycle 9 -- Modélisation des actions mécaniques \\
Ch. 1 : Modélisation des actions de pesanteur et de contact sans frottement -- \xxactivite%
}


\setcounter{secnumdepth}{5}
%---------------------------------------------------------------------------


\begin{document}
%\chapterimage{png/Fond_Cin}
\input{style/new_pagegarde}
\vspace{7cm}
\pagestyle{fancy}
\thispagestyle{plain}


\def\columnseprulecolor{\color{ocre}}
\setlength{\columnseprule}{0.4pt} 

\begin{multicols}{2}
\subsection*{Exercice 1 : Portion de disque}
\begin{flushright}
\textit{D'après Agati et al., Mécanique du solide, Applications industrielles, Dunod.}
\end{flushright}
%\setcounter{subparagraph}{0}

%\begin{minipage}[c]{.3\linewidth}
\begin{center}
\includegraphics[width=.7\linewidth]{images/disque}
\end{center}
%\end{minipage}\hfill
%\begin{minipage}[c]{.65\linewidth}
Soit une plaque homogène, d'épaisseur négligeable, ayant la forme d'un quart de cercle de rayon $r$. Le matériau est de masse surfacique $\mu$.

\subparagraph{}
\textit{Déterminer la masse du solide.}

\subparagraph{}
\textit{Déterminer la position du centre de gravité.}

\subparagraph{}
\textit{Déterminer la torseur des actions de pesanteur.}

%\ifthenelse{\boolean{prof}}{
%\begin{corrige}
%OG = \dfrac{4\sqrt{2}}{3\pi} r 
%\end{corrige}
%}{}




\subsection*{Exercice 2 -- Portion de cylindre}
\setcounter{exo}{0}
\begin{center}
\includegraphics[width=\linewidth]{images/portioncylindre}
\end{center}

Soit une portion cylindrique de masse volumique $\mu$ et de secteur angulaire 2$\theta$.

\subparagraph{}
\textit{Paramétrer le solide.}
\subparagraph{}
\textit{Déterminer la masse du solide.}

\subparagraph{}
\textit{Déterminer la position du centre de gravité.}

\subparagraph{}
\textit{Déterminer la torseur des actions de pesanteur.}




\subsection*{Exercice 3 : plaque}
\setcounter{exo}{0}
\begin{flushright}
\textit{D'après Agati et al., Mécanique du solide, Applications industrielles, Dunod.}
\end{flushright}
\setcounter{subparagraph}{0}

\begin{center}
\includegraphics[width=.65\linewidth]{images/plaque}
\end{center}

Soit une plaque homogène, d'épaisseur négligeable, ayant la forme ci-contre. Le matériau est de masse surfacique $\mu$.

\subparagraph{}
\textit{Déterminer la masse du solide.}

\subparagraph{}
\textit{Déterminer la position du centre de gravité.}

\subparagraph{}
\textit{Déterminer la torseur des actions de pesanteur.}


\subsection*{Exercice 4 : Barrage de la Tamise}
\setcounter{exo}{0}

Le Thames Barrier est un barrage spectaculaire conçu pour protéger la ville de Londres des marrées exceptionnellement élevées qui peuvent remonter de la mer. Sa construction terminée en 1982 a nécessité 51 000 tonnes d'acier et 210 000 $m^3$ de béton, ce qui en fait le second barrage mobile le plus grand du monde.



\begin{center}
\includegraphics[width=.9\linewidth]{images/fig1}
\end{center}


La structure s'étend sur 520 mètres de large et est constituée de 10 portes de forme de secteur angulaire de 20 mètres de haut. Chaque porte est totalement effacée dans un berceau en béton coulé au fond de la rivière. En cas de montée des eaux, les portes pivotent en position verticale actionnées par une machine hydraulique.

\begin{center}
\includegraphics[width=.9\linewidth]{images/fig2}
\end{center}




L'objectif est de déterminer la position du centre de gravité de la porte qui est une structure creuse en tôle épaisse et donc on donne le modèle ci contre.

\textbf{Données :}
\begin{itemize}
\item longueur porte : $L=58\;m$
\item Rayon : $R=12,4\;m$
\item épaisseur tôle : $e=0,05\;m$ (considérée négligeable devant $R$)
\item masse volumique de la porte : $\rho=7\,700 \; kg/m^3$
\item $\alpha=\pi/3$
\end{itemize}

\begin{center}
\includegraphics[width=.9\linewidth]{images/fig3}
\end{center}


\subparagraph{}
\textit{Déterminer les coordonnées du centre de gravité de la porte.}
\subparagraph{}
\textit{Déterminer la torseur des actions de pesanteur.}


\subsection*{Exercice 5 : Modélisation des actions mécaniques de contact sur un palier lisse}

On souhaite déterminer le modèle global des actions mécaniques de contact sur un palier lisse, composant technologique pour le guidage en rotation.

On donne le modèle local :
\begin{itemize}
\item les surfaces de contact sont limitées par un demi cylindre de longueur $L$ et de rayon $R$;
\item entre les surfaces de contact, la pression $p$ est uniforme sur chaque élément $dS$ situé autour du point $M$.
\end{itemize}

\begin{center}
\includegraphics[width=.9\linewidth]{images/fig5}
\end{center}

\begin{center}
\includegraphics[width=.75\linewidth]{images/fig6_bis}
\end{center}


\setcounter{exo}{0}
\subparagraph{}
\textit{Déterminer le modèle global de l'action mécanique de l'arbre 2 sur le bâti 1 sous la forme d'un torseur exprimé au point $O$.}
\ifprof
\begin{corrige}

Exprimons le torseur des actions mécaniques sous sa forme locale en un point $M$ : 

$$
\torseurl{d\vectf{2}{1}}{d\vectm{M}{2}{1}=\vect{0}}{M}
$$

La forme globale au point O est alors donnée par :

$$
\torseurstat{T}{2}{1} = \torseurl{\vectf{2}{1} = \int d\vectf{2}{1}}{\vectm{M}{2}{1} = \int d\vectm{M}{2}{1}= \int \vect{OM}\wedge d\vectf{2}{1}}{M}
$$

\vspace{.5cm}

\textbf{Calculons $\vectf{2}{1}$.}

$$
\vectf{2}{1} = \int d\vectf{2}{1} = \iint p \vect{-r} dS = -p \iint  \vect{r} dS
= -p \iint  \left(\cos\theta\vect{x}+\sin\theta\vect{y} \right) dS $$

$$
\vectf{2}{1}
= -p \int\limits_{-L/2}^{L/2} \int\limits_{-\pi/2}^{\pi/2}   \left(\cos\theta\vect{x}+\sin\theta\vect{y} \right) Rd\theta dz
= -p R L \int\limits_{-\pi/2}^{\pi/2}   \left(\cos\theta\vect{x}+\sin\theta\vect{y} \right) d\theta 
$$
$$
\vectf{2}{1}
= -p R L \left(\int\limits_{-\pi/2}^{\pi/2}   \cos\theta\vect{x}d\theta + \int\limits_{-\pi/2}^{\pi/2} \sin\theta\vect{y} d\theta \right)
= -p R L \left(\left[\sin\theta \right]_{-\pi/2}^{\pi/2}\vect{x}
+\left[ -\cos\theta\right]_{-\pi/2}^{\pi/2}\vect{y}
\right)
$$

$$
\vectf{2}{1}
= -p R L \left(2\vect{x}
+0\vect{y}
\right) = -2pRL\vect{x}
$$

$2RL$ est appelée surface projetée du cylindre. Elle correspond au produit du diamètre par sa longueur.

\vspace{.5cm}

\textbf{Calculons $\vectm{M}{2}{1}$.}
$$
\vectm{M}{2}{1} = \int d\vectm{M}{2}{1}= \int \vect{OM}\wedge d\vectf{2}{1}
$$

$$
\vectm{M}{2}{1} = -p \iint R\vect{r} \wedge \vect{r}dS = \vect{0}
$$

Au final, 
$$
\torseurstat{T}{2}{1} = \torseurl{\vectf{2}{1} = -2pRL\vect{x}}{\vectm{M}{2}{1} =  \vect{0}}{M}
$$



\textbf{Calculer $\vectf{2}{1}$ lorsque la pression est de la forme : $p(\theta)=p_0\cos\theta$ pour $\theta\in[-\pi/2,\pi/2]$.}

Dans ce cas : 
$$
\vectf{2}{1} = \int d\vectf{2}{1} = \iint p(\theta) \vect{-r} dS 
= - p_0R\iint\cos\theta  \left(\cos\theta\vect{x}+\sin\theta\vect{y} \right)  d\theta dz$$

$$
\vectf{2}{1} 
= - p_0 L R\int\limits_{-\pi/2}^{\pi/2}\cos\theta  \left(\cos\theta\vect{x}+\sin\theta\vect{y} \right)  d\theta$$

$$
\int\limits_{-\pi/2}^{\pi/2}\cos^2\theta  d\theta = \dfrac{\pi}{2}
\quad 
\text{et}
\quad
\int\limits_{-\pi/2}^{\pi/2}\cos\theta \sin\theta  d\theta = 0
$$
Au final :
$$
\vectf{2}{1} 
= - p_0 L R \dfrac{\pi}{2}\vect{x}$$
\end{corrige}
\else
\fi

\end{multicols}
\end{document}


